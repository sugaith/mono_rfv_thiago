% !TeX encoding = UTF-8

\chapter{REVISÃO BIBLIOGRÁFICA}\label{ch:rev-bibs}

Alguns trabalhos serviram como ajuda e inspiração para este estudo. Porém durante o período de busca por bibliografias .... \autoref{ch:introducao}.

De acordo com \citeonline{credito-bancario}, um dado se transforma em informação ....

Em seu estudo, \citeonline{credito-bancario} aborda duas técnicas ...

...

O reconhecimento de padrões permite ....  \cite{logs-web}. Para o desenvolvimento ....


\section{AQUISIÇÃO E PROCESSAMENTO DE IMAGENS E VÍDEO}\label{sec:processamento_imagens}


\subsection{IMAGEM E VÍDEO DIGITAL}\label{subsec:imagem}

\subsection{FORMATO PNG}\label{subsec:png}

\subsection{AQUISIÇÃO DE IMAGEM}\label{subsec:aquisicao_img}

\subsection{AQUISIÇÃO DE IMAGEM EM VÍDEO}\label{subsec:aquisicao_video}

\subsection{PROCESSAMENTO}\label{subsec:processamento}

\subsubsection{FILTROS}\label{subsubsec:filtros}

\subsubsection{ESCALONAMENTO}\label{subsubsec:escalonamento}

\subsubsection{EQUALIZAÇÃO}\label{subsubsec:equalizacao}

\subsubsection{SEGMENTAÇÃO}\label{subsubsec:segmentacao}

\subsubsection{ILUMINAÇÃO}\label{subsubsec:iluminacao}

\subsection{ARMAZENAMENTO}\label{subsubsec:armazenamento}

\subsection{EXIBIÇÃO}\label{subsubsec:exibicao}




\section{DETECÇÃO DE FACES COM ELEMENTOS HAAR}\label{sec:detecao_faces}

Comprimentos de o


\subsection{ELEMENTOS HAAR (\textit{HAAR FEATURES}) }\label{subsubsec:elem_haar}

%FALAR SOBRE HAAR WAVE LET

\subsection{ALGORÍTIMO VIOLA-JONES (\textit{HAAR CASCADES CLASSIFIER}) }\label{subsubsec:violajones}





\section{RECONHECIMENTO DE FACES COM ANALISE DE COMPONENTE PRINCIPAL (ACP) }\label{sec:recog_faces}


\subsection{ANALISE DE COMPONENTE PRINCIPAL (ACP)}\label{subsec:acp}


\subsection{ACP para \textit{EngenFaces}}\label{subsec:acp}

\subsection{TREINAMENTO}\label{subsec:treiamento}

\subsection{RECONHECIMENTO}\label{subsec:reconhecimento}



























